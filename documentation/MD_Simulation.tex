\documentclass{article}

\usepackage{fancyhdr}
\usepackage{extramarks}
\usepackage[usenames,dvipsnames]{color}
\usepackage{graphicx}
\usepackage{listings}
\usepackage{courier}
\usepackage{hyperref}

\def\versionnumber{1.0.0}

% Margins
\topmargin=-0.45in
\evensidemargin=0in
\oddsidemargin=0in
\textwidth=6.5in
\textheight=9.0in
\headsep=0.25in

\pagestyle{fancy}
\lhead{Compact Cori Python MD Simulation Documentation}
\chead{}
\rhead{\today}
\cfoot{}
\lfoot{National Energy Research Scientific Computing Center}
\rfoot{\thepage}
\renewcommand\headrulewidth{0.4pt} % Size of the header rule
\renewcommand\footrulewidth{0.4pt} % Size of the footer rule

\setlength\parindent{0pt} % Removes all indentation from paragraphs

\definecolor{MyDarkGreen}{rgb}{0.0,0.4,0.0} % This is the color used for comments
\lstloadlanguages{Verilog}
\lstset{language=Python,
        frame=single,
        basicstyle=\small\ttfamily,
        keywordstyle=[1]\color{Blue}\bf,
        keywordstyle=[2]\color{Purple},
        keywordstyle=[3]\color{Red},
        identifierstyle=,
        commentstyle=\usefont{T1}{pcr}{m}{sl}\color{MyDarkGreen}\small,
        stringstyle=\color{Purple},
        showstringspaces=false,
        tabsize=2,
        morecomment=[l][\color{Blue}]{...},
        numbers=left,
        numberstyle=\tiny,
        breaklines=true
}

\begin{document}
\title{\textsc{Compact Cori Python MD Simulation Documentation}\\ Version \versionnumber}
\author {Nick Fong\\nbfong[at]lbl.gov}
\maketitle
\tableofcontents

\section{Introduction}

\section{Files}
\subsection{particle\_simulation.py}
This file contains the logic that links the simulation together.  Particles and
Partitions are created, and timesteps are performed.  The RESTful API endpoints
are created here.
\subsection{util.py}
This file contains utility functions that may be used from any other file after
\texttt{util.py} is imported.  This helps keep things DRY.
\subsection{params.py}
This file contains global parameters for the simulation to avoid having to pass
around multiple parameters for each method.  Note that when a variable in
\texttt{params.py} is updated in \texttt{Partition}, for example, the change is
reflected in \texttt{Particle} as well.  However, if a change is made on MPI
task 3, the change is not reflected on MPI task 2.
\subsection{Particle.py}
This class holds attributes of each particle in the simulation.  In addition,
the class contains methods to update the Particle's attributes based on
neighboring Particles' properties.
\subsection{Partition.py}
A Partition helps abstract what one MPI slave task works on.  Each task has its
own partition that corresponds to a volume of the simulation.  Particles are
owned by a Partition, and are passed between Partitions.
\section{Classes}
\section{MPI}
\section{API}

\section{Acknowledgements}
This work was supported by the Director, Office of Science, Division of
Mathematical, Information, and Computational Sciences of the U.S. Department of
Energy under contract DE-AC02-05CH11231.\\

This research used resources of the National Energy Research Scientific
Computing Center, which is supported by the Office of Science of the U.S.
Department of Energy.

\end{document}


